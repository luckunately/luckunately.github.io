%%%%%%%%%%%%%%%%%
% This is an sample CV template created using altacv.cls
% (v1.7, 9 August 2023) written by LianTze Lim (liantze@gmail.com). Compiles with pdfLaTeX, XeLaTeX and LuaLaTeX.
%
%% It may be distributed and/or modified under the
%% conditions of the LaTeX Project Public License, either version 1.3
%% of this license or (at your option) any later version.
%% The latest version of this license is in
%%    http://www.latex-project.org/lppl.txt
%% and version 1.3 or later is part of all distributions of LaTeX
%% version 2003/12/01 or later.
%%%%%%%%%%%%%%%%

%% Use the "normalphoto" option if you want a normal photo instead of cropped to a circle
% \documentclass[10pt,a4paper,normalphoto]{altacv}

\documentclass[10pt,a4paper,ragged2e,withhyper]{altacv}
%% AltaCV uses the fontawesome5 and packages.
%% See http://texdoc.net/pkg/fontawesome5 for full list of symbols.

% Change the page layout if you need to
\geometry{left=1.25cm,right=1.25cm,top=1.5cm,bottom=1.5cm,columnsep=1.2cm}

% The paracol package lets you typeset columns of text in parallel
\usepackage{paracol}
\usepackage{fancyhdr}

% Change the font if you want to, depending on whether
% you're using pdflatex or xelatex/lualatex
% WHEN COMPILING WITH XELATEX PLEASE USE
% xelatex -shell-escape -output-driver="xdvipdfmx -z 0" sample.tex
\ifxetexorluatex
  % If using xelatex or lualatex:
  \setmainfont{Roboto Slab}
  \setsansfont{Lato}
  \renewcommand{\familydefault}{\sfdefault}
\else
  % If using pdflatex:
  \usepackage[rm]{roboto}
  \usepackage[defaultsans]{lato}
  % \usepackage{sourcesanspro}
  \renewcommand{\familydefault}{\sfdefault}
\fi

% Change the colours if you want to
\definecolor{SlateGrey}{HTML}{2E2E2E}
\definecolor{LightGrey}{HTML}{666666}
\definecolor{DarkPastelRed}{HTML}{450808}
\definecolor{PastelRed}{HTML}{8F0D0D}
\definecolor{GoldenEarth}{HTML}{E7D192}
\colorlet{name}{black}
\colorlet{tagline}{PastelRed}
\colorlet{heading}{DarkPastelRed}
\colorlet{headingrule}{GoldenEarth}
\colorlet{subheading}{PastelRed}
\colorlet{accent}{PastelRed}
\colorlet{emphasis}{SlateGrey}
\colorlet{body}{LightGrey}

% Change some fonts, if necessary
\renewcommand{\namefont}{\Huge\rmfamily\bfseries}
\renewcommand{\personalinfofont}{\footnotesize}
\renewcommand{\cvsectionfont}{\LARGE\rmfamily\bfseries}
\renewcommand{\cvsubsectionfont}{\large\bfseries}


% Change the bullets for itemize and rating marker
% for \cvskill if you want to
\renewcommand{\cvItemMarker}{{\small\textbullet}}
\renewcommand{\cvRatingMarker}{\faCircle}
% ...and the markers for the date/location for \cvevent
% \renewcommand{\cvDateMarker}{\faCalendar*[regular]}
% \renewcommand{\cvLocationMarker}{\faMapMarker*}


% If your CV/résumé is in a language other than English,
% then you probably want to change these so that when you
% copy-paste from the PDF or run pdftotext, the location
% and date marker icons for \cvevent will paste as correct
% translations. For example Spanish:
% \renewcommand{\locationname}{Ubicación}
% \renewcommand{\datename}{Fecha}


%% Use (and optionally edit if necessary) this .tex if you
%% want to use an author-year reference style like APA(6)
%% for your publication list
% \input{pubs-authoryear.tex}

%% Use (and optionally edit if necessary) this .tex if you
%% want an originally numerical reference style like IEEE
%% for your publication list
\input{pubs-num.tex}

%% sample.bib contains your publications
\addbibresource{sample.bib}

\begin{document}
\name{Xingyu (Tom) Wang}
\tagline{Bachelor of Applied Science in Computer Engineering}
%% You can add multiple photos on the left or right
\photoR{3cm}{ubc-logo}
% \photoL{2.5cm}{Yacht_High,Suitcase_High}

% \rhead{\parbox[][2.5cm][t]{\textwidth}{
% \includegraphics[height=1.15cm]{ubc-logo.png}
% }}

\personalinfo{%
  % Not all of these are required!
  \email{fortily@student.ubc.ca}
  \email{tomxingyuwang@gmail.com}
  \phone{(+1)604-388-5164}
  \homepage{luckunately.github.io}
  \linkedin{www.linkedin.com/in/tom-wang-554904220/}
  \github{https://github.com/luckunately}
  %% You can add your own arbitrary detail with
  %% \printinfo{symbol}{detail}[optional hyperlink prefix]
  % \printinfo{\faPaw}{Hey ho!}[https://example.com/]

  %% Or you can declare your own field with
  %% \NewInfoFiled{fieldname}{symbol}[optional hyperlink prefix] and use it:
  % \NewInfoField{gitlab}{\faGitlab}[https://gitlab.com/]
  % \gitlab{your_id}
  %%
  %% For services and platforms like Mastodon where there isn't a
  %% straightforward relation between the user ID/nickname and the hyperlink,
  %% you can use \printinfo directly e.g.
  % \printinfo{\faMastodon}{@username@instace}[https://instance.url/@username]
  %% But if you absolutely want to create new dedicated info fields for
  %% such platforms, then use \NewInfoField* with a star:
  % \NewInfoField*{mastodon}{\faMastodon}
  %% then you can use \mastodon, with TWO arguments where the 2nd argument is
  %% the full hyperlink.
  % \mastodon{@username@instance}{https://instance.url/@username}
}

\makecvheader
%% Depending on your tastes, you may want to make fonts of itemize environments slightly smaller
% \AtBeginEnvironment{itemize}{\small}

%% Set the left/right column width ratio to 6:4.
\columnratio{0.7}

% Start a 2-column paracol. Both the left and right columns will automatically
% break across pages if things get too long.
\begin{paracol}{2}
\cvsection{Experience}

\cvevent{FPGA Soft IP Engineering Intern}{Altera}{May 2025 -- August 2026}{Toronto, ON}
\begin{itemize}
  \item Working on Test Engine IP in the HBM (High Bandwidth Memory) subsystem
  \item Develop Feature, verify functionality and reduce resource usage of the Test Engine IP
\end{itemize}

\divider

\cvevent{Student Research Asistant}{UBC}{April 2024 -- April 2025}{Vancouver, BC}
\begin{itemize}
\item Investigated supervised learning methods (LSTM, Transformer, etc.) for page prefetching using collected traces; achieved better results than heuristic algorithms (LEAP) on various workloads, with ongoing challenges in deployment and inference time.
\item Supervision under: Shaurya Patel, Prof. Alexandra Fedorova.
\end{itemize}

\cvsection{Projects}

\cvevent{Evaluating Cache Scheduling Strategies for vLLM Inference}{}{January 2025 -- April 2025}{}
\begin{itemize}
  \item Experiment OS cache prefetching strategies to for vLLM inference.
  \item Explore adaptive watermark tuning techniques to optimize memory usage and scheduling.
\end{itemize}

\divider

\cvevent{Capstone: Reinforcement Learning with SVT-AV1 Codec}{}{January 2025 -- August 2025}{}
\begin{itemize}
  \item Used reinforcement learning to improve AV1 Codec constant bitrate mode by assigning Quantization Parameter (QP) offsets to superblocks within a frame, given a frame-level QP.
  \item Built an RL environment by exposing the C program API, enabling per-video optimization; generalization across different videos remains challenging.
\end{itemize}

\divider

\cvevent{Microsystem Design with Microprocessor}{}{Jan 2024 -- April 2024}{}
\begin{itemize}
\item Build {memory, data bus, various I/O} around a M68K CPU on FPGA. Interact with CPU using embedded C programming
\item Implemented components including {DRAM controller, Cache Controller, SPI, Canbus, I2C, ADC/DAC}, and {Simple RTOS} usage with multi-threading and priority interrupts.
\item Integrate the above components with VGA and Voice modules, and map addresses accordingly both in RTL design and C programming to produce a Tetris game with the M68K CPU
\end{itemize}

% \divider

% \cvevent{ECC Performance Analysis on FPGA}{}{Mar 2024 -- April 2024}{}
% \begin{itemize}
%     \item RTL design of simple decoder and encoder for both Hamming code and LDPC code on FPGA.
%     \item Analyze and compare performance on decode/encode cycle, combinational logic length, maximal frequency, gate usage, efficiency, and ease of use on DE1-SOC FPGA board.
% \end{itemize}

\medskip

% \cvsection{A Day of My Life}

% % Adapted from @Jake's answer from http://tex.stackexchange.com/a/82729/226
% % \wheelchart{outer radius}{inner radius}{
% % comma-separated list of value/text width/color/detail}
% \wheelchart{1.5cm}{0.5cm}{%
%   6/8em/accent!30/{Sleep,\\beautiful sleep},
%   3/8em/accent!40/Hopeful novelist by night,
%   8/8em/accent!60/Daytime job,
%   2/10em/accent/Sports and relaxation,
%   5/6em/accent!20/Spending time with family
% }

% % use ONLY \newpage if you want to force a page break for
% % ONLY the current column
% \newpage

% \cvsection{Publications}

% %% Specify your last name(s) and first name(s) as given in the .bib to automatically bold your own name in the publications list.
% %% One caveat: You need to write \bibnamedelima where there's a space in your name for this to work properly; or write \bibnamedelimi if you use initials in the .bib
% %% You can specify multiple names, especially if you have changed your name or if you need to highlight multiple authors.
% \mynames{Lim/Lian\bibnamedelima Tze,
%   Wong/Lian\bibnamedelima Tze,
%   Lim/Tracy,
%   Lim/L.\bibnamedelimi T.}
% %% MAKE SURE THERE IS NO SPACE AFTER THE FINAL NAME IN YOUR \mynames LIST

% \nocite{*}

% \printbibliography[heading=pubtype,title={\printinfo{\faBook}{Books}},type=book]

% \divider

% \printbibliography[heading=pubtype,title={\printinfo{\faFile*[regular]}{Journal Articles}},type=article]

% \divider

% \printbibliography[heading=pubtype,title={\printinfo{\faUsers}{Conference Proceedings}},type=inproceedings]

%% Switch to the right column. This will now automatically move to the second
%% page if the content is too long.
\switchcolumn

% \cvsection{My Life Philosophy}

% \begin{quote}
% ``Something smart or heartfelt, preferably in one sentence.''
% \end{quote}

\cvsection{Awards}

\cvachievement{\faTrophy}{NSERC Awards}{Natrual Sciences and Engineering Research Council of Canada Undergraduate Student Research Award (USRA) for May 2024 - August 2024}

% \divider

% \cvachievement{\faHeartbeat}{Dean's Honors List}{Through out the academic journey in UBC}

% \divider

% \cvachievement{\faHeartbeat}{Another achievement}{more details about it of course}

\cvsection{Skills}

% C, Python, SystemVerilog
\cvtag{C} \cvtag{Python} \cvtag{SystemVerilog} \\
% ML, Pytorch
\cvtag{Machine Learning} \cvtag{Pytorch} \\
% OS, QEMU
\cvtag{Operating Systems} \cvtag{QEMU} \\
% Computer Architecture, Digital Logic Design
\cvtag{Computer Architecture} \\ \cvtag{Digital Logic Design} \\
% VLSI, Microprocessor Design
\cvtag{VLSI} \cvtag{Microprocessor Design} \\

% % Java, Python, C, C++
% \cvtag{Java} \cvtag{Python} \cvtag{C} \cvtag{C++} \\
% % System Verilog, Assembly
% \cvtag{System Verilog} \cvtag{Assembly} \\
% % Bash, Makefile
% \cvtag{Bash} \cvtag{Makefile} 
% % latex, Markdown
% \cvtag{LaTeX} \cvtag{Markdown} \\

% \divider\smallskip \\
% % FPGA, RTL, Quartus
% \cvtag{FPGA} \cvtag{RTL} \cvtag{Quartus} \\
% % Microprocessor, Embedded Systems
% \cvtag{Microprocessor} \cvtag{Embedded Systems} \\
% % Memory Hierarchies, Cache, Page Prefetching
% \cvtag{Computer Architecture} \cvtag{Prefetching} \\
% % Software Development, Debugging, Software Hardware Co-design
% \cvtag{Software Development} \cvtag{Debugging} \cvtag{Software Hardware Co-design} \\
% % Machine Learning, LSTM, Transformer
% \cvtag{Machine Learning} \cvtag{LSTM} \\
% % Git, Linux
% \cvtag{Git} \cvtag{Linux}


% \cvsection{Languages}

% \cvskill{English}{5}
% \divider

% \cvskill{Spanish}{4}
% \divider

% \cvskill{German}{3.5} %% Supports X.5 values.

%% Yeah I didn't spend too much time making all the
%% spacing consistent... sorry. Use \smallskip, \medskip,
%% \bigskip, \vspace etc to make adjustments.
% \medskip

\cvsection{Education}

% \cvevent{Ph.D.\ in Your Discipline}{Your University}{Sept 2002 -- June 2006}{}
% Thesis title: Wonderful Research

% \divider

% \cvevent{M.Sc.\ in Your Discipline}{Your University}{Sept 2001 -- June 2002}{}

% \divider

\cvevent{BASC.\ in Computer Engineering}{University of British Columbia}{Sept 2021 -- Aug 2026}{}
\textbf{CGPA:} 87\% \\
\textbf{Upper-level (3rd year+) courses:} 89\% \\
\textbf{Affiliations:} Systopia Lab \\

\textbf{Course Highlights:}
\begin{itemize}
  \item \textbf{Computer Systems:} Computer Architecture, Digital \& Microsystem Design, Computing Systems, VLSI
  \item \textbf{Software:} Software Development, Data Structures \& Algorithms, Operating Systems
  \item \textbf{Other:} Machine Learning, Error Control Coding, Abstract Math
\end{itemize}
% \divider

% \cvsection{Referees}

% % \cvref{name}{email}{mailing address}
% \cvref{Prof.\ Alpha Beta}{Institute}{a.beta@university.edu}
% {Address Line 1\\Address line 2}

% \divider

% \cvref{Prof.\ Gamma Delta}{Institute}{g.delta@university.edu}
% {Address Line 1\\Address line 2}


\end{paracol}


\end{document}
